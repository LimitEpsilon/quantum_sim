%! TEX program = xelatex
\documentclass{article}
\usepackage{geometry}
\geometry{
 a4paper,
 total={170mm,257mm},
 left=20mm,
 top=20mm,
}
\usepackage{setspace} % setstretch

\usepackage{graphicx}
\usepackage{csquotes}

\usepackage{tabularray}
\UseTblrLibrary{booktabs}
\UseTblrLibrary{counter}

\usepackage{biblatex}
\addbibresource{citations.bib}

%%% Math settings
\usepackage{amssymb,amsmath,mathtools} % Before unicode-math
\usepackage[math-style=TeX,bold-style=TeX]{unicode-math}
\newtheorem{clm}{Claim}[section]

%%% Font settings
\setmainfont{Libertinus Serif}
\setsansfont{Libertinus Sans}[Scale=MatchUppercase]
\setmonofont{JuliaMono}[Scale=MatchLowercase]
\setmathfont{Libertinus Math}
\setmathfont{TeX Gyre Pagella Math}[range={\lbrace,\rbrace},Scale=1.1]

%%% PL constructs
\usepackage{ebproof}
\ebproofset{left label template=\textsc{[\inserttext]}}

% For simplebnf
\newfontfamily{\fallbackfont}{EB Garamond}
\DeclareTextFontCommand{\textfallback}{\fallbackfont}
\usepackage{newunicodechar}
\newunicodechar{⩴}{\textfallback{⩴}}

\usepackage{simplebnf}[2022/05/08]
\RenewDocumentCommand\SimpleBNFDefEq{}{\ensuremath{⩴}}

% because of simplebnf
\newcommand*\vbar{|}
\newcommand*{\finto}{\xrightarrow{\text{\textrm{fin}}}}
\newcommand*{\istype}{\mathrel{⩴}}
\newcommand*{\ortype}{\mathrel{|}}

% for complement
\newcommand{\loverbar}[1]{\mkern 1.5mu\overline{\mkern-1.5mu#1\mkern-1.5mu}\mkern 1.5mu}

\newcommand*{\Reanalyze}{\textit{ReAnalyze}}

\title{Constructing Set Constraints for ReScript}
\author{Joonhyup Lee}
\date{}
\begin{document}
\maketitle

\section{Calculations}
\begin{align*}
  i\hbar\partial_{t}\rho &= \hbar\omega[a^{\dag}a, \rho]+\hbar\chi^{(2)}[\frac{\alpha^{*}e^{2i\omega t}a^{2}-\alpha e^{-2i\omega t}(a^{\dag})^{2}}{2i}, \rho]\\
                       &= \hbar\omega[a^{\dag}a, \rho]+\frac{\hbar\chi^{(2)}}{2i}([\alpha^{*}e^{2i\omega t}a^{2}, \rho]-[\alpha e^{-2i\omega t}(a^{\dag})^{2}, \rho])
\end{align*}

Let $\rho=\exp(-i\omega ta^{\dag}a)\tilde{\rho}\exp(i\omega ta^{\dag}a)$, then
\[\partial_{t}\tilde{\rho}=-\frac{\chi^{(2)}}{2}(\alpha^{*}[a^{2},\tilde{\rho}]-\alpha[(a^{\dag})^{2},\tilde{\rho}])\]

We want to look at the time evolution of the Wigner quasiprobability distribution. 
The Von Neumann equation above can be translated into a partial differential equation for the Wigner function by the following rules:

\begin{align*}
  a\rho&\leftrightarrow (z+\frac{1}{2}\partial_{z^{*}})W & z &\leftrightarrow q+ip\\
  a^{\dag}\rho&\leftrightarrow (z^{*}-\frac{1}{2}\partial_{z})W & z^{*} &\leftrightarrow q-ip\\
  \rho a&\leftrightarrow (z-\frac{1}{2}\partial_{z^{*}})W & \partial_{z} &\leftrightarrow \frac{1}{2}(\partial_{q}-i\partial_{p})\\
  \rho a^{\dag}&\leftrightarrow (z^{*}-\frac{1}{2}\partial_{z})W & \partial_{z^{*}} &\leftrightarrow \frac{1}{2}(\partial_{q}+i\partial_{p})
\end{align*}

Then the Von Neumann equation turns into
\[\partial_{t}W=-\frac{\chi^{(2)}}{2}(2\alpha z^{*}\partial_{z}W+2\alpha^{*}z\partial_{z^{*}}W)\]
\printbibliography
\end{document}
%%% Local Variables: 
%%% coding: utf-8
%%% mode: latex
%%% TeX-engine: xetex
%%% End:
