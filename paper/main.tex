%! TEX program = xelatex
\documentclass{article}
\usepackage{geometry}
\geometry{
 a4paper,
 total={170mm,257mm},
 left=20mm,
 top=20mm,
}
\usepackage{setspace} % setstretch

\usepackage{graphicx}
\usepackage{csquotes}

\usepackage{tabularray}
\UseTblrLibrary{booktabs}
\UseTblrLibrary{counter}

\usepackage{biblatex}
\addbibresource{citations.bib}

%%% Math settings
\usepackage{amssymb,amsmath,mathtools} % Before unicode-math
\usepackage[math-style=TeX,bold-style=TeX]{unicode-math}
\newtheorem{clm}{Claim}[section]

%%% Font settings
\setmainfont{Libertinus Serif}
\setsansfont{Libertinus Sans}[Scale=MatchUppercase]
\setmonofont{JuliaMono}[Scale=MatchLowercase]
\setmathfont{Libertinus Math}
\setmathfont{TeX Gyre Pagella Math}[range={\lbrace,\rbrace},Scale=1.1]

% for complement
\newcommand{\loverbar}[1]{\mkern 1.5mu\overline{\mkern-1.5mu#1\mkern-1.5mu}\mkern 1.5mu}

% rotated wigner function
\newcommand*{\wigner}{W(\omega z, \omega^{-1}z^{*};t)}

\title{Analysis of Squeezed Vacuum States of Light by Means of Wigner Functions}
\author{Joonhyup Lee}
\date{}
\begin{document}
\maketitle

\section{Second-order Squeezing}
\begin{align*}
  i\hbar\partial_{t}\rho & = \hbar\omega[a^{\dag}a, \rho]+\hbar\chi^{(2)}\left[\frac{\alpha^{*}e^{2i\omega t}a^{2}-\alpha e^{-2i\omega t}(a^{\dag})^{2}}{2i}, \rho\right] \\
                         & = \hbar\omega[a^{\dag}a, \rho]+\frac{\hbar\chi^{(2)}}{2i}([\alpha^{*}e^{2i\omega t}a^{2}, \rho]-[\alpha e^{-2i\omega t}(a^{\dag})^{2}, \rho])
\end{align*}

Let $\rho=\exp(-i\omega ta^{\dag}a)\tilde{\rho}\exp(i\omega ta^{\dag}a)$, then
\[\partial_{t}\tilde{\rho}=-\frac{\chi^{(2)}}{2}(\alpha^{*}[a^{2},\tilde{\rho}]-\alpha[(a^{\dag})^{2},\tilde{\rho}])\]

We want to look at the time evolution of the Wigner quasiprobability distribution.
The Von Neumann equation above can be translated into a partial differential equation for the Wigner function by the following rules:

\begin{align*}
  a\rho         & \leftrightarrow \left(z+\frac{1}{2}\partial_{z^{*}}\right)W & z                & \leftrightarrow q+ip                                    \\
  a^{\dag}\rho  & \leftrightarrow \left(z^{*}-\frac{1}{2}\partial_{z}\right)W & z^{*}            & \leftrightarrow q-ip                                    \\
  \rho a        & \leftrightarrow \left(z-\frac{1}{2}\partial_{z^{*}}\right)W & \partial_{z}     & \leftrightarrow \frac{1}{2}(\partial_{q}-i\partial_{p}) \\
  \rho a^{\dag} & \leftrightarrow \left(z^{*}-\frac{1}{2}\partial_{z}\right)W & \partial_{z^{*}} & \leftrightarrow \frac{1}{2}(\partial_{q}+i\partial_{p})
\end{align*}

Then the Von Neumann equation turns into
\[\partial_{t}W=-\frac{\chi^{(2)}}{2}(2\alpha z^{*}\partial_{z}W+2\alpha^{*}z\partial_{z^{*}}W)\]

Now we want to evaluate how the Wigner function evolves in time given the distribution at time $0$.
That is, we want to calculate the trajectory $(q(t), p(t))$ that satisfies
\[W(q(t),p(t);t)=W(q(0),p(0);0)\]
for all time $t$.

Differentiating both sides by $t$, we get, by the chain rule,
\[\frac{dz}{dt}\partial_{z}W+\frac{dz^{*}}{dt}\partial_{z^{*}}W+\partial_{t}W=0\]
if we view $W$ as a function of $z=q+ip$ and $z^{*}=q-ip$.

By the time evolution equation, this equation can be summarized into
\[\left(\frac{dz}{dt}-\chi^{(2)}\alpha z^{*}\right)\partial_{z}W+\left(\frac{dz^{*}}{dt}-\chi^{(2)}\alpha^{*} z\right)\partial_{z^{*}}W=0\]

Therefore, if the trajectory $z(t)=q(t)+ip(t)$ satisfies
\[\frac{dz}{dt}=\chi^{(2)}\alpha z^{*}\]
then for any initial condition we can determine the Wigner function at time $t$.

Separating into real and imaginary parts, we have
\[
  \frac{d}{dt}
  \begin{bmatrix}
    q \\
    p
  \end{bmatrix}
  = \chi^{(2)}
  \begin{bmatrix}
    \Re(\alpha) & \Im(\alpha)  \\
    \Im(\alpha) & -\Re(\alpha)
  \end{bmatrix}
  \begin{bmatrix}
    q \\
    p
  \end{bmatrix}
\]
and solving this linear ode results in
\[
  \begin{bmatrix}
    q \\
    p
  \end{bmatrix}
  =
  R_{\theta/2}
  \begin{bmatrix}
    e^{\chi^{(2)}rt} & 0                 \\
    0                & e^{-\chi^{(2)}rt}
  \end{bmatrix}
  R_{-\theta/2}
  \begin{bmatrix}
    q(0) \\
    p(0)
  \end{bmatrix}
\]
when $\alpha=re^{i\theta}$ and $R_{\theta/2}$ is the rotation matrix by $\theta/2$.
Thus, we have shown that for \textit{any} initial distribution the distribution is squeezed by
$e^{\chi^{2}r}$.
\section{Third-order Squeezing}
\begin{align*}
  i\hbar\partial_{t}\rho & = \hbar\omega[a^{\dag}a, \rho]+\hbar\chi^{(3)}\left[\frac{\alpha^{*}e^{3i\omega t}a^{3}-\alpha e^{-3i\omega t}(a^{\dag})^{3}}{2i}, \rho\right] \\
                         & = \hbar\omega[a^{\dag}a, \rho]+\frac{\hbar\chi^{(3)}}{2i}([\alpha^{*}e^{3i\omega t}a^{3}, \rho]-[\alpha e^{-3i\omega t}(a^{\dag})^{3}, \rho])
\end{align*}

Let $\rho=\exp(-i\omega ta^{\dag}a)\tilde{\rho}\exp(i\omega ta^{\dag}a)$, then
\[\partial_{t}\tilde{\rho}=-\frac{\chi^{(3)}}{2}(\alpha^{*}[a^{3},\tilde{\rho}]-\alpha[(a^{\dag})^{3},\tilde{\rho}])\]
Now, the time evolution of the Wigner function is given by
\[\partial_{t}W=-\frac{\chi^{(3)}}{2}\left(\alpha^{*}\left(3z^{2}\partial_{z^{*}}W+\frac{1}{4}\partial_{z^{*}}^{3}W\right)+\alpha\left(3(z^{*})^{2}\partial_{z}W+\frac{1}{4}\partial_{z}^{3}W\right)\right)\]

We consider the rotated distribution $\tilde{W}(z,z^{*};t)\coloneq W(\omega z, \omega^{-1}z^{*};t)$ when $\omega=\exp(i2\pi/3)$ is the third root of unity.
We will show that $\tilde{W}$ also satisfies the above equation.
Note that
\begin{align*}
  \partial_{t}\tilde{W}(z,z^{*};t)     & =\partial_{t}\wigner               \\
  \partial_{z}\tilde{W}(z,z^{*};t)     & =\omega\partial_{z}\wigner         \\
  \partial_{z}^{2}\tilde{W}(z,z^{*};t) & =\omega^{2}\partial_{z}^{2}\wigner \\
  \partial_{z}^{3}\tilde{W}(z,z^{*};t) & =\partial_{3}^{3}\wigner
\end{align*}
by the chain rule.

Then we have:
\begin{align*}
  \partial_{t}\tilde{W}(z,z^{*};t) & =\partial_{t}\wigner                                                                                                                                      \\
                                   & =-\frac{\chi^{(3)}}{2}\left( \alpha^{*}\left( 3(\omega z)^{2}\partial_{z^{*}}\wigner+\frac{1}{4}\partial_{z^{*}}^{3}\wigner\right)\right.                 \\
                                   & \left.+\alpha\left(3(\omega^{-1}z^{*})^{2}\partial_{z}\wigner+\frac{1}{4}\partial_{z}^{3}\wigner\right)\right)                                            \\
                                   & =-\frac{\chi^{(3)}}{2}\left( \alpha^{*}\left(3z^{2}\partial_{z^{*}}\tilde{W}(z,z^{*};t)+\frac{1}{2}\partial_{z^{*}}^{3}\tilde{W}(z,z^{*};t)\right)\right. \\
                                   & \left.+\alpha\left( 3(z^{*})^{2}\partial_{z}\tilde{W}(z,z^{*};t)+\frac{1}{4}\partial_{z^{*}}^{3}\tilde{W}(z,z^{*};t)\right)\right)
\end{align*}

Thus, we know that
\[\partial_{t}(W-\tilde{W})=-\frac{\chi^{(3)}}{2}\left(\alpha^{*}\left(3z^{2}\partial_{z^{*}}+\frac{1}{4}\partial_{z^{*}}^{3}\right)+\alpha\left(3(z^{*})^{2}\partial_{z}+\frac{1}{4}\partial_{z}^{3}\right)\right)(W-\tilde{W})\]
If we have that for time $0$, $(W-\tilde{W})(q,p;0)=0$, that is, if the probability distribution is symmetric with respect to $120^{\circ}$ rotations, then for the rest of the time, $W-\tilde{W}$ stays $0$. That is, the symmetry is preserved in time.
This can explain why the squeezed vacuum subject to the third order generation process displays a Wigner quasiprobability distribution that is shaped like an equilateral triangle.

\printbibliography
\end{document}
%%% Local Variables: 
%%% coding: utf-8
%%% mode: latex
%%% TeX-engine: xetex
%%% End:
